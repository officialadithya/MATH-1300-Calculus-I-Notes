
\begin{multicols}{2}
      \setlength{\parindent}{0pt}
      \footnotesize{

\textbf{Chapter} \ref{chapter:funcmod}, \textsc{Page} \pageref{chapter:funcmod} \\
\textsc{Definition} \ref{def:functions}, \textsc{Page} \pageref{def:functions} \textit{Functions} \\
\textsc{Definition} \ref{def:graphs}, \textsc{Page} \pageref{def:graphs} \textit{Functions' Graphs} \\
\textsc{Definition} \ref{def:arrowdiagrams}, \textsc{Page} \pageref{def:arrowdiagrams} \textit{Functions' Arrow Diagrams} \\
\textsc{Theorem} \ref{thm:verticallinetest}, \textsc{Page} \pageref{thm:verticallinetest} \textit{The Vertical Line Test} \\
\textsc{Definition} \ref{def:piecewisefunc}, \textsc{Page} \pageref{def:piecewisefunc} \textit{Piecewise Functions} \\
\textsc{Definition} \ref{def:evenoddfunc}, \textsc{Page} \pageref{def:evenoddfunc} \textit{Even and Odd Functions} \\
\textsc{Definition} \ref{def:incdecfunc}, \textsc{Page} \pageref{def:incdecfunc} \textit{Increasing and Decreasing Functions} \\
\textsc{Definition} \ref{def:mathematicalmodels}, \textsc{Page} \pageref{def:mathematicalmodels} \textit{Mathematical Models} \\
\textsc{Definition} \ref{def:linfunc}, \textsc{Page} \pageref{def:linfunc} \textit{Linear Functions} \\
\textsc{Definition} \ref{def:polynomials}, \textsc{Page} \pageref{def:polynomials} \textit{Polynomials} \\
\textsc{Definition} \ref{def:powerfunc}, \textsc{Page} \pageref{def:powerfunc} \textit{Power Functions} \\
\textsc{Definition} \ref{def:ratfunc}, \textsc{Page} \pageref{def:ratfunc} \textit{Rational Functions} \\
\textsc{Definition} \ref{def:algfunc}, \textsc{Page} \pageref{def:algfunc} \textit{Algebraic Functions} \\
\textsc{Definition} \ref{def:trigfunc}, \textsc{Page} \pageref{def:trigfunc} \textit{Trigonometric Functions} \\
\textsc{Definition} \ref{def:expfunc}, \textsc{Page} \pageref{def:expfunc} \textit{Exponential Functions} \\
\textsc{Definition} \ref{def:logfunc}, \textsc{Page} \pageref{def:logfunc} \textit{Logarithmic Functions} \\
\textsc{Theorem} \ref{thm:functrans}, \textsc{Page} \pageref{thm:functrans} \textit{Function Transformations} \\
\textsc{Definition} \ref{def:algcombfunc}, \textsc{Page} \pageref{def:algcombfunc} \textit{Algebraic Combinations of Functions} \\
\textsc{Definition} \ref{def:compfunc}, \textsc{Page} \pageref{def:compfunc} \textit{Compositions of Functions} \\
\textbf{Chapter} \ref{chapter:limder}, \textsc{Page} \pageref{chapter:limder} \\
\textsc{Definition} \ref{def:limits}, \textsc{Page} \pageref{def:limits} \textit{Limits} \\
\textsc{Definition} \ref{def:onesidedlimits}, \textsc{Page} \pageref{def:onesidedlimits} \textit{One-Sided Limits} \\
\textsc{Theorem} \ref{thm:limonelim}, \textsc{Page} \pageref{thm:limonelim} \textit{The Relationship Between Limits and One-Sided Limits} \\
\textsc{Theorem} \ref{thm:limlaws}, \textsc{Page} \pageref{thm:limlaws} \textit{Limit Laws} \\
\textsc{Theorem} \ref{thm:derlimlaws}, \textsc{Page} \pageref{thm:derlimlaws} \textit{Derived Limit Laws} \\
\textsc{Theorem} \ref{thm:dirsub}, \textsc{Page} \pageref{thm:dirsub} \textit{Direct Substitution} \\
\textsc{Theorem} \ref{thm:limequivtwofunc}, \textsc{Page} \pageref{thm:limequivtwofunc} \textit{Limit Equivalence of Two Functions} \\
\textsc{Theorem} \ref{thm:limineq}, \textsc{Page} \pageref{thm:limineq} \textit{Limits Preserve Inequalities} \\
\textsc{Theorem} \ref{thm:squeezethm}, \textsc{Page} \pageref{thm:squeezethm} \textit{The Squeeze Theorem} \\
\textsc{Theorem} \ref{thm:specialtriglims}, \textsc{Page} \pageref{thm:specialtriglims} \textit{Useful Trigonometric Limits} \\
\textsc{Definition} \ref{def:continuity}, \textsc{Page} \pageref{def:continuity} \textit{Continuity} \\
\textsc{Definition} \ref{def:onesidedcont}, \textsc{Page} \pageref{def:onesidedcont} \textit{One-Sided Continuity} \\
\textsc{Definition} \ref{def:intervalcont}, \textsc{Page} \pageref{def:intervalcont} \textit{Continuity on an Interval} \\
\textsc{Theorem} \ref{thm:continuityofnewfunc}, \textsc{Page} \pageref{thm:continuityofnewfunc} \textit{Continuity of New Functions} \\
\textsc{Theorem} \ref{thm:contfunc}, \textsc{Page} \pageref{thm:contfunc} \textit{Continuous Functions} \\
\textsc{Theorem} \ref{thm:limcomp}, \textsc{Page} \pageref{thm:limcomp} \textit{Limit of a Composition} \\
\textsc{Theorem} \ref{thm:compcontcont}, \textsc{Page} \pageref{thm:compcontcont} \textit{Composition of Continuous Functions is Continuous} \\
\textsc{Theorem} \ref{thm:ivt}, \textsc{Page} \pageref{thm:ivt} \textit{The Intermediate Value Theorem} \\
\textsc{Definition} \ref{def:inflim}, \textsc{Page} \pageref{def:inflim} \textit{Infinite Limits} \\
\textsc{Definition} \ref{def:vertasymp}, \textsc{Page} \pageref{def:vertasymp} \textit{Vertical Asymptotes} \\
\textbf{Chapter} \ref{chapter:diffrules}, \textsc{Page} \pageref{chapter:diffrules} \\
\textbf{Chapter} \ref{chapter:appdiff}, \textsc{Page} \pageref{chapter:appdiff} \\
\textbf{Chapter} \ref{chapter:integrals}, \textsc{Page} \pageref{chapter:integrals} \\

      }
\end{multicols}

