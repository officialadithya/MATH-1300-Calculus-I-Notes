\section{Week 1: January 16 -- January 20}

    \subsection{Four Ways to Represent a Function}
		Consider the following definition.
		\begin{definition}{\Stop\,\,Functions}{functions}

			A function \(f\) is a rule that assigns to each element in a set \(D\) exactly one element, called \(f(x)\) in a set \(E\).

		\end{definition}
		\vphantom
		\\
		\\
		In the context of Definition \ref{def:functions}, the set \(D\) is the domain of \(f\). The range of \(f\) is the set of all possible values of \(f(x)\) for all \(x\) in the domain.
		\\
		\\
		Functions can, for the purposes of this course, be represented in the following ways:
		\begin{enumerate}
			\item Verbally, with a description in words.
			\item Numerically, with a table of values.
			\item Visually, with a graph.
			\item Algebraically, with an explicit formula.
		\end{enumerate}
		\pagebreak
		\vphantom
		\\
		\\
		It is often useful to use graphs or arrow diagrams to visualize functions. Consider the following definitions.
		\begin{definition}{\Stop\,\,Functions' Graphs}{graphs}
			
			If \(f\) is a function with domain \(D\), the graph of \(f\) consists of all points \((x,f(x))\) in the \(xy\) plane. Equivalently, the graph of \(f\) is the set of ordered pairs
			\begin{equation*}
				\{(x,f(x)):x\in D\}.
			\end{equation*}

		\end{definition}
		\begin{definition}{\Stop\,\,Functions' Arrow Diagrams}{arrowdiagrams}
			
			If \(f\) is a function with domain \(D\), the arrow diagram of \(f\) consists of a selection of points \((x,f(x))\) organized in the following manner.
			\begin{center}
				\begin{tikzpicture}[
					>=stealth,
					bullet/.style={
					fill=black,
					circle,
					minimum width=1pt,
					inner sep=1pt
					},
					projection/.style={
					->,
					thick,
					shorten <=2pt,
					shorten >=2pt
					},
					every fit/.style={
					ellipse,
					draw,
					text width=50pt,
					align = center
					},
					scale=0.8
					]
					\node[bullet,label=above:\(D\)] (D) at (0,-1) {};
					\node[bullet,label=above:\(\)] (E) at (4,-1) {};
					\node[bullet,label=below:\(\vdots\)] (dEND) at (0,-5) {};
					\node[bullet,label=below:\(\vdots\)] (eEND) at (4,-5) {};
					\foreach \y/\l in {1/x_1,2/x_2,3/x_3,4/x_4,5/x_5}
					\node[bullet,label=left:$\l$] (d\y) at (0,-1*\y) {};
				
					\foreach \y/\l in {1/f(x_1),2/f(x_2),3/f(x_3),4/f(x_4),5/f(x_5)}
					\node[bullet,label=right:$\l$] (e\y) at (4,-1*\y) {};
				
					\node[draw,fit=(d1) (d2) (d3) (d4) (d5) (dEND), minimum width=2cm] {} ;
					\node[draw,fit=(e1) (e2) (e3) (e4) (e5) (eEND), minimum width=2cm] {} ;
				
					\draw[projection] (d1) -- (e1);
					\draw[projection] (d2) -- (e2);
					\draw[projection] (d3) -- (e3);
					\draw[projection] (d4) -- (e4);
					\draw[projection] (d5) -- (e5);
					
				\end{tikzpicture}
			\end{center}
			\vphantom
			\\
			\\
			Often, it is not practical, or impossible, to add all elements of \(D\) to the left ellipse, so a useful selection of elements of \(D\) is used instead.
			
		\end{definition}
		\pagebreak
		\vphantom
		\\
		\\
		Consider the following examples.
		\begin{example}{\Difficulty\,\Difficulty\,\,Creating a Graph 1}{creategraph1}
			
			Graph the function \(f(x)=x^3\) on the interval \([-2,2]\).
			\\
			\\
			Consider the following graph.
			\begin{center}
				\begin{tikzpicture}
					\begin{axis} [def]
						\addplot [domain=-2:2, restrict y to domain=-2:2] {x^3};
					\end{axis}
				\end{tikzpicture}
			\end{center}

		\end{example}
		\begin{example}{\Difficulty\,\Difficulty\,\,Creating a Graph 2}{creategraph2}

			Graph the function \(f(x)=\frac{1}{x}\) on the interval \([-2,2]\).
			\\
			\\
			Consider the following graph.
			\begin{center}
				\begin{tikzpicture}
					\begin{axis} [def]
						\addplot [domain=-2:2, restrict y to domain=-2:2] {1/x};
					\end{axis}
				\end{tikzpicture}
			\end{center}
			
		\end{example}
		\pagebreak
		\begin{example}{\Difficulty\,\Difficulty\,\,Creating a Graph 3}{creategraph3}

			Graph the function \(f(x)=\tan x\) on the interval \([-\pi,\pi]\).
			\\
			\\
			Consider the following graph.
			\begin{center}
				\begin{tikzpicture}
					\begin{axis} [def]
						\addplot [domain=-pi:pi, restrict y to domain=-4:4] {tan(deg(x))};
					\end{axis}
				\end{tikzpicture}
			\end{center}
			
		\end{example}
		\begin{example}{\Difficulty\,\Difficulty\,\,Creating an Arrow Diagram 1}{creatarrowdiagram1}

			Create an arrow diagram for the function \(f(x)=x^3\) with the integers \(\{-2,-1,0,1,2\}\).
			\\
			\\
			Consider the following diagram.
			\begin{center}
				\begin{tikzpicture}[
					>=stealth,
					bullet/.style={
					fill=black,
					circle,
					minimum width=1pt,
					inner sep=1pt
					},
					projection/.style={
					->,
					thick,
					shorten <=2pt,
					shorten >=2pt
					},
					every fit/.style={
					ellipse,
					draw,
					text width=50pt,
					align = center
					},
					scale=0.8
					]
					\node[bullet,label=above:\(D\)] (D) at (0,-1) {};
					\node[bullet,label=above:\(\)] (E) at (4,-1) {};
					\node[bullet,label=below:\(\vdots\)] (dEND) at (0,-5) {};
					\node[bullet,label=below:\(\vdots\)] (eEND) at (4,-5) {};
					\foreach \y/\l in {1/-2,2/-1,3/0,4/1,5/2}
					\node[bullet,label=left:$\l$] (d\y) at (0,-1*\y) {};
				
					\foreach \y/\l in {1/-8,2/-1,3/0,4/1,5/8}
					\node[bullet,label=right:$\l$] (e\y) at (4,-1*\y) {};
				
					\node[draw,fit=(d1) (d2) (d3) (d4) (d5) (dEND), minimum width=2cm] {} ;
					\node[draw,fit=(e1) (e2) (e3) (e4) (e5) (eEND), minimum width=2cm] {} ;
				
					\draw[projection] (d1) -- (e1);
					\draw[projection] (d2) -- (e2);
					\draw[projection] (d3) -- (e3);
					\draw[projection] (d4) -- (e4);
					\draw[projection] (d5) -- (e5);
					
				\end{tikzpicture}
			\end{center}
			
		\end{example}
		\pagebreak
		\begin{example}{\Difficulty\,\Difficulty\,\,Creating an Arrow Diagram 2}{creatarrowdiagram2}

			Create an arrow diagram for the function \(f(x)=\frac{1}{x}\) with the integers \(\{-2,-1,1,2\}\).
			\\
			\\
			Consider the following diagram.
			\begin{center}
				\begin{tikzpicture}[
					>=stealth,
					bullet/.style={
					fill=black,
					circle,
					minimum width=1pt,
					inner sep=1pt
					},
					projection/.style={
					->,
					thick,
					shorten <=2pt,
					shorten >=2pt
					},
					every fit/.style={
					ellipse,
					draw,
					text width=50pt,
					align = center
					},
					scale=0.8
					]
					\node[bullet,label=above:\(D\)] (D) at (0,-1) {};
					\node[bullet,label=above:\(\)] (E) at (4,-1) {};
					\node[bullet,label=below:\(\vdots\)] (dEND) at (0,-4) {};
					\node[bullet,label=below:\(\vdots\)] (eEND) at (4,-4) {};
					\foreach \y/\l in {1/-2,2/-1,3/1,4/2}
					\node[bullet,label=left:$\l$] (d\y) at (0,-1*\y) {};
				
					\foreach \y/\l in {1/-\frac{1}{2},2/-1,3/1,4/\frac{1}{2}}
					\node[bullet,label=right:$\l$] (e\y) at (4,-1*\y) {};
				
					\node[draw,fit=(d1) (d2) (d3) (d4) (dEND), minimum width=2cm] {} ;
					\node[draw,fit=(e1) (e2) (e3) (e4) (eEND), minimum width=2cm] {} ;
				
					\draw[projection] (d1) -- (e1);
					\draw[projection] (d2) -- (e2);
					\draw[projection] (d3) -- (e3);
					\draw[projection] (d4) -- (e4);
					
				\end{tikzpicture}
			\end{center}
			
		\end{example}
		\begin{example}{\Difficulty\,\Difficulty\,\,Creating an Arrow Diagram 3}{creatarrowdiagram3}

			Create an arrow diagram for the function \(f(x)=\frac{1}{x}\) with the values \(\left\{0,\frac{\pi}{6},\frac{\pi}{4},\frac{\pi}{3}\right\}\).
			\\
			\\
			Consider the following arrow diagram.
			\begin{center}
				\begin{tikzpicture}[
					>=stealth,
					bullet/.style={
					fill=black,
					circle,
					minimum width=1pt,
					inner sep=1pt
					},
					projection/.style={
					->,
					thick,
					shorten <=2pt,
					shorten >=2pt
					},
					every fit/.style={
					ellipse,
					draw,
					text width=50pt,
					align = center
					},
					scale=0.8
					]
					\node[bullet,label=above:\(D\)] (D) at (0,-1) {};
					\node[bullet,label=above:\(\)] (E) at (4,-1) {};
					\node[bullet,label=below:\(\vdots\)] (dEND) at (0,-4) {};
					\node[bullet,label=below:\(\vdots\)] (eEND) at (4,-4) {};
					\foreach \y/\l in {1/0,2/\frac{\pi}{6},3/\frac{\pi}{4},4/\frac{\pi}{3}}
					\node[bullet,label=left:$\l$] (d\y) at (0,-1*\y) {};
				
					\foreach \y/\l in {1/0,2/\frac{\sqrt{3}}{3},3/1,4/\sqrt{3}}
					\node[bullet,label=right:$\l$] (e\y) at (4,-1*\y) {};
				
					\node[draw,fit=(d1) (d2) (d3) (d4) (dEND), minimum width=2cm] {} ;
					\node[draw,fit=(e1) (e2) (e3) (e4) (eEND), minimum width=2cm] {} ;
				
					\draw[projection] (d1) -- (e1);
					\draw[projection] (d2) -- (e2);
					\draw[projection] (d3) -- (e3);
					\draw[projection] (d4) -- (e4);
					
				\end{tikzpicture}
			\end{center}

		\end{example}
		\vphantom
		\\
		\\
		Often, given a graph, we must be able to tell whether the curve is a function. Consider the following theorem.
		\begin{theorem}{\Stop\,\,The Vertical Line Test}{verticallinetest}
			
			A curve in the \(xy\) plane is the graph of a function if and only if no verticl line intersects the curve more than once.

		\end{theorem}
		\pagebreak
		\vphantom
		\\
		\\
		Consider the following examples.
		\begin{example}{\Difficulty\,\Difficulty\,\,Is the Curve the Graph of a Function 1}{curvefunc1}
			
			Is the graph below a function or not?
			\begin{center}
			\begin{tikzpicture}
				\begin{axis} [def]
					\addplot [domain=-2:2] ({x^3-3*x},{3*x^2-9}); 
				\end{axis}
			\end{tikzpicture}
			\end{center}
			\vphantom
			\\
			\\
			No. The graph does not correspond to a function.

		\end{example}
		\begin{example}{\Difficulty\,\Difficulty\,\,Is the Curve the Graph of a Function 2}{curvefunc2}

		Is the graph below a function or not?
			\begin{center}
				\begin{tikzpicture}
					\begin{axis} [def]
						\addplot [domain=-pi:pi] {sin(deg(x))}; 
					\end{axis}
				\end{tikzpicture}
			\end{center}
			\vphantom
			\\
			\\
			Yes. The graph corresponds to a function.
		
		\end{example}
		\pagebreak
		\vphantom
		\\
		\\
		Consider the following definition.
		\begin{definition}{\Stop\,\,Piecewise Functions}{piecewisefunc}

			Piecewise functions are those that have multiple assignment rules for \(f(x)\) depending on the interval \(x\) is in.
		
		\end{definition}
		\vphantom
		\\
		\\
		It is often useful to graph piecewise functions. Consider the following examples.
		\begin{example}{\Difficulty\,\Difficulty\,\,Graphing a Piecewise Function 1}{graphpiecewise1}
		
			Graph the function
			\begin{equation*}
				f(x)=\begin{cases}
					-x & x < 0 \\
					x & x \geq 0
				\end{cases}
			\end{equation*}
			on the interval \([-3,3]\).
			\\
			\\
			We see that \(f(x)\) is really the familiar absolute value function, \(|x|\). We graph the result of applying the rule for \(f(x)\) on its corresponding inequality, which denotes the interval for which the rule is valid. Consider the following graph.
			\begin{center}
				\begin{tikzpicture}
					\begin{axis} [def]
						\addplot [domain=-3:0] {-x}; 
						\addplot [domain=0:3] {x}; 
					\end{axis}
				\end{tikzpicture}
			\end{center}

		\end{example}
		\pagebreak
		\begin{example}{\Difficulty\,\Difficulty\,\,Graphing a Piecewise Function 2}{graphpiecewise2}
			
			Graph the function
			\begin{equation*}
				f(x)=\begin{cases}
					x-1 & x < 0 \\
					x^2 & x \geq 0
				\end{cases}
			\end{equation*}
			on the interval \([-2,2]\).
			\\
			\\
			We graph the result of applying the rule for \(f(x)\) on its corresponding inequality, which denotes the interval for which the rule is valid. Consider the following graph.
			\begin{center}
				\begin{tikzpicture}
					\begin{axis} [def]
						\addplot [domain=-2:0] {x-1};
						\addplot [domain=0:2] {x^2};
						\addplot [only marks, mark=o] table {
						0 -1
						};
						\addplot [only marks] table {
						0 0
						};
					\end{axis}
				\end{tikzpicture}
			\end{center}

		\end{example}
		\vphantom
		\\
		\\
		We now turn to even and odd functions. Consider the following definition.
		\begin{definition}{\Stop\,\,Even and Odd Functions}{evenoddfunc}

			A function \(f(x)\) is even if and only if \(f(-x)=f(x)\) and odd if and only if \(f(-x)=-f(x)\). Functions that don't have either property are called neither.
			
		\end{definition}
		\pagebreak
		\vphantom
		\\
		\\
		Consider the following examples.
		\begin{example}{\Difficulty\,\Difficulty\,\,Is it Even, Odd, or Neither 1}{evenoddneither1}
			
			Is \(f(x)=\sin x\) even, odd, or neither?
			\\
			\\
			Consider, for some real numbers \(a\) and \(b\) such that \(a-b=-1\),
			\begin{align*}
				f(-x)=\sin(-x)&=\sin(ax-bx) \\
				&=\sin(ax)\cos(bx)-\sin(bx)\cos(ax) \\
				&=-(-\sin(ax)\cos(bx)+\sin(bx)\cos(ax)) \\
				&=-(\sin(bx)\cos(ax)-\sin(ax)\cos(bx)) \\
				&=-\sin(bx-ax) \\
				&=-\sin((b-a)x) \\
				&=-\sin x=-f(x).
			\end{align*}
			Therefore, \(f(x)\) is odd.

		\end{example}
		\begin{example}{\Difficulty\,\Difficulty\,\,Is it Even, Odd, or Neither 2}{evenoddneither2}
			
			Is \(f(x)=1-x^2\) even, odd, or neither?
			\\
			\\
			Consider
			\begin{align*}
				f(-x)&=1-(-x)^2 \\
				&=1-x^2=f(x).
			\end{align*}
			Therefore, \(f(x)\) is even.

		\end{example}
		\begin{example}{\Difficulty\,\Difficulty\,\,Is it Even, Odd, or Neither 3}{evenoddneither3}

			Is \(f(x)=e^{2x}\) even, odd, or neither?
			\\
			\\
			Consider
			\begin{align*}
				f(-x)&=e^{-2x} \\
				&=\frac{1}{e^{2x}}.
			\end{align*}
			Therefore, \(f(x)\) is neither even nor odd.
			
		\end{example}
		\vphantom
		\\
		\\
		Graphically, even functions are symmetric about the \(y\) axis, and odd functions are symmetric about the origin.
		\pagebreak
		\vphantom
		\\
		\\
		Consider the following definition.
		\begin{definition}{\Stop\,\,Increasing and Decreasing Functions}{incdecfunc}

			A function \(f\) is increasing on an interval \(I\) if and only if \(f(x_1)<f(x_2)\) whenever \(x_1<x_2\) for \(x_1,x_2\in I\). Similarly, \(f\) is decreasing on an interval \(I\) if and only if \(f(x_1)>f(x_2)\) whenever \(x_1<x_2\) for \(x_1,x_2\in I\).
			
		\end{definition}

	\pagebreak

	\subsection{Mathematical Models: A Catalog of Essential Functions}

		Consider the following definitions.
		\begin{definition}{\Stop\,\,Mathematical Models}{mathematicalmodels}
			
			A mathematical model is a mathematical description of a real-world phenomenon that is used for analysis of the phenomenon. Mathematical models are never fully accurate and seek to balance simplification to permit calculation with accuracy to provide valuable information.

		\end{definition}
		\vphantom
		\\
		\\
		We will now define various functions that will allow us to utilize mathematical modelling.
		\begin{definition}{Stop\,\,Linear Functions}{linfunc}
			
			A function \(f(x)\) is linear if and only if
			\begin{equation*}
				f(x)=mx+b
			\end{equation*}
			for real numbers \(m\) and \(b\). Graphically, \(m\) is the slope of \(f(x)\) and \(b\) is the \(y\) intercept.

		\end{definition}
		\begin{definition}{\Stop\,\,Polynomials}{polynomials}

			A function \(f(x)\) is a polynomial if and only if
			\begin{equation*}
				f(x)=a_nx^n+a_{n-1}x^{n-1}+\cdots+a_2x^2+a_1x+a_0
			\end{equation*}
			for real numbers \(a_1,\ldots,a_n\) and a nonnegative integer \(n\). If the leading coefficient, \(a_n\) is nonzero, the degree of the polynomial is \(n\).
			
		\end{definition}
		\vphantom
		\\
		\\
		If the degree of a polynomial is \(2\), it is called quadratic and has the form
		\begin{equation*}
			Q(x)=ax^2+bx+c.
		\end{equation*}
		If the degree of a polynomial is \(3\), it is called cubic and has the form
		\begin{equation*}
			C(x)=ax^3+bx^2+cx+d.
		\end{equation*}
		\begin{definition}{\Stop\,\,Power Functions}{powerfunc}

			A function \(f(x)\) is a power function if and only if
			\begin{equation*}
				f(x)=x^a
			\end{equation*}
			for some real number \(a\). Let \(n\) be a positive integer. If \(a=n\), \(f(x)\) is a polynomial. If \(a=\frac{1}{n}\), \(f(x)=x^\frac{1}{n}=\sqrt[n]{x}\) is a root function. If \(a=-1\), \(f(x)\) is the reciprocal function \(\frac{1}{x}\).
			
		\end{definition}
		\begin{definition}{\Stop\,\,Rational Functions}{ratfunc}

			A function \(f(x)\) is a rational function if and only if 
			\begin{equation*}
				f(x)=\frac{P(x)}{Q(x)}
			\end{equation*}
			for polynomials \(P\) and \(Q\). The function \(f(x)\) is defined for all \(x\) where \(Q(x)\neq0\).
			
		\end{definition}
		\begin{definition}{\Stop\,\,Algebraic Functions}{algfunc}

			A function \(f(x)\) is a rational function if and only if \(f(x)\) can be constructed using only addition, subtraction, multiplication, division, and taking roots to manipulate polynomials.
			
		\end{definition}
		\begin{definition}{\Stop\,\,Trigonometric Functions}{trigfunc}

			A function \(f(x)\) is a trigonometric function if and only if \(f(x)\) involves the sine, cosine, tangent, cosecant, secant, or cotangent functions. For this text, radians will always be used in lieu of degrees, unless explicitly stated. 
			
		\end{definition}
		\begin{definition}{\Stop\,\,Exponential Functions}{expfunc}

			A function \(f(x)\) is a rational function if and only if
			\begin{equation*}
				f(x)=a^x
			\end{equation*}
			for some real number \(a\).
			
		\end{definition}
		\begin{definition}{\Stop\,\,Logarithmic Functions}{logfunc}

			A function \(f(x)\) is a rational function if and only if
			\begin{equation*}
				f(x)=\log_ax
			\end{equation*}
			for some real number \(a\).
			
		\end{definition}
		\vphantom
		\\
		\\
		A familiarity with the above definitions is crucial to understanding this text.
	
	\pagebreak
	
	\subsection{New Functions from Old Functions}

		Consider the following function transformations.
		\begin{theorem}{\Stop\,\,Function Transformations}{functrans}
			
			Let \(f\) be a function and \(c>0\). Then,
			\begin{enumerate}
				\item To find the graph of \(f(x)+c\), shift the graph of \(f(x)\) \(c\) units upward.
				\item To find the graph of \(f(x)-c\), shift the graph of \(f(x)\) \(c\) units downward.
				\item To find the graph of \(f(x-c)\), shift the graph of \(f(x)\) \(c\) units rightward.
				\item To find the graph of \(f(x+c)\), shift the graph of \(f(x)\) \(c\) units leftward.
			\end{enumerate}
			\vphantom
			\\
			\\
			Now, let \(c>1\). Then,
			\begin{enumerate}
				\item To find the graph of \(cf(x)\), stretch the graph of \(f(x)\) vertically by a factor of \(c\).
				\item To find the graph of \(\frac{1}{c}f(x)\), compress the graph of \(f(x)\) vertically by a factor of \(c\).
				\item To find the graph of \(f\left(\frac{1}{c}x\right)\), stretch the graph of \(f(x)\) horizontally by a factor of \(c\).
				\item To find the graph of \(f(cx)\), compress the graph of \(f(x)\) horizontally by a factor of \(c\).
				\item To find the graph of \(-f(x)\), reflect the graph of \(f(x)\) about the \(x\) axis.
				\item To find the graph of \(f(-x)\), reflect the graph of \(f(x)\) about the \(y\) axis.
			\end{enumerate}

		\end{theorem}
		\pagebreak
		\vphantom
		\\
		\\
		Theorem \ref{thm:functrans} is extremely useful for graphing functions when it is easy to rewrite a function in terms of a transformation of a simpler function. Consider the following example.
		\begin{example}{\Difficulty\,\Difficulty\,\,Using Function Transformations to Graph a Function}{functransgraph}

			Graph the function
			\begin{equation*}
				f(x)=x^2+2x+3
			\end{equation*}
			on the interval \([-4,2]\).
			\\
			\\
			Notice that \(f(x)=x^2+2x+3=(x^2+2x+1)+2=(x+1)^2+2\). Therefore, the graph of \(f(x)\) is the graph of \(g(x)=x^2\) shifted one unit leftward and two units upward. Consider the following graph.
			\begin{center}
				\begin{tikzpicture}
					\begin{axis} [def]
						\addplot [domain=-4:2] {x^2+2*x+3};
					\end{axis}
				\end{tikzpicture}
			\end{center}
			
		\end{example}
		\vphantom
		\\
		\\
		We can also define certain combinations of functions.
		\begin{definition}{\Stop\,\,Algebraic Combinations of Functions}{algcombfunc}
			
			Let \(f\) and \(g\) be functions. Then,
			\begin{equation*}
				(f+g)(x)=f(x)+g(x),\quad (f-g)(x)=f(x)-g(x)
			\end{equation*}
			and
			\begin{equation*}
				(fg)(x)=f(x)g(x),\quad \left(\frac{f}{g}\right)(x)=\frac{f(x)}{g(x)}.
			\end{equation*}
			In all cases, the domain of the combination is the set of values that are in both the domains of \(f\) and \(g\).

		\end{definition}
		\begin{definition}{\Stop\,\,Compositions of Functions}{compfunc}
			
			Let \(f\) and \(g\) be functions. Then,
			\begin{equation*}
				(f\circ g)(x)=f(g(x)).
			\end{equation*}
			The domain of the composition is the set of all \(x\) in the domain of \(g\) such that \(g(x)\) is in the domain of \(f\).

		\end{definition}
		\vphantom
		\\
		\\
		Consider the following examples.
		\begin{example}{\Difficulty\,\Difficulty\,\,A Sum of Functions}{funcsum}

			Let \(f(x)=\sin^2x\) and \(g(x)=\cos^2x\). Find \((f+g)(x)\).
			\\
			\\
			We see that
			\begin{align*}
				(f+g)(x)&=f(x)+g(x) \\
				&=\sin^2x+\cos^2x \\
				&=1.
			\end{align*}
		\end{example}
		\begin{example}{\Difficulty\,\Difficulty\,\,A Product of Functions}{prodfunc}

			Let \(f(x)=2\sin x\) and \(g(x)=\cos x\). Find \((fg)(x)\).
			\\
			\\
			We see that
			\begin{align*}
				(fg)(x)&=f(x)g(x) \\
				&=2\sin x\cos x \\
				&=\sin(2x).
			\end{align*}
		\end{example}
		\begin{example}{\Difficulty\,\Difficulty\,\,A Composition of Functions 1}{compfunc1}

			Let \(f(x)=2\sin x\) and \(g(x)=e^{2x+3}\). Find \((f\circ g)(x)\).
			\\
			\\
			We see that
			\begin{align*}
				(f\circ g)(x)&=f(g(x)) \\
				&=2\sin(e^{2x+3}).
			\end{align*}
		\end{example}
		\begin{example}{\Difficulty\,\Difficulty\,\,A Composition of Functions 2}{compfunc2}

			If \(h(x)=(x+3\cos x)^2\). Find functions \(f\) and \(g\) such that \((f\circ g)(x)=h(x)\).
			\\
			\\
			We see that \(f(x)=x^2\) and \(g(x)=x+3\cos x\) satisfy the desired property.

		\end{example}